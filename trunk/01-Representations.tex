\documentclass[MachineLearning]{subfiles}
\begin{document}

\section{Representations}

% summarizes lecture 1 & 2
%author: Benjamin Ellenberger


\section{Measurements and Data}
How to represent data is a crucial part to successful learning. The concrete choice of representation ('features') Learning methods expect standardized representations of data (e.g. Points in vector spaces, nodes in a graph, similarity matrices ...)
\subsection{Taxonomy of data}
\textbf{Goal:} We like to represent objects/items of interest and
characterize them according to their typical
patterns for detection, classification, abstraction (compression), $\ldots$.

\begin{itemize}
\item Object space: Design/Configuration/ Object space \(\mathcal{O}\)
\item Measurement: Measurement X maps an object into a domain 
\(X: \mathcal{O}^{(1)}\ldots\mathcal{O}^{(R)}\rightarrow \K\) and represents an object in a data space.
\item Data
\begin{itemize}
\item \textbf{Monadic data} \(\mathcal{O} \rightarrow \R^{d}\): An indivisible, impenetrable unit of data such as temperature, water depth, pressure, intensity etc..
\item \textbf{Dyadic data} \(\mathcal{O}^{(1)}\times \mathcal{O}^{(2)} \rightarrow \R\): Data consisting of two monadic datasets of different object spaces. This can be \(\{users\} \times \{websites\}\) etc..
\item \textbf{Polyadic data} \(\mathcal{O}^{(1)}\times \mathcal{O}^{(2)} \times \ldots \times \mathcal{O}^{(R)} \rightarrow \R\): Data consisting of multiple monadic datasets of different object spaces. 
\item \textbf{Pairwise data} \(\mathcal{O} \times \mathcal{O} \rightarrow \R\): Data that comes in pairs of the same object space such as \(\{proteins\} \times \{proteins\}\) comparisons.
\end{itemize}
\end{itemize}

The choice of features X is an engineering problem. We must find a tradeoff between underfitting and overfitting the data.
\begin{enumerate}
\item Ideally as informative as possible about label Y
\item Must consider cost of acquiring / computing them
\item Poor choice of features \(\rightarrow\) no luck with learning!
\end{enumerate}
\subsection{Patterns}
\subsection{Data Types, Transformations, Scale}
\begin{itemize}
\item \textbf{Nominal or categorical scale}: Qualitative, but without quantitative measurements, e.g. binary scale \(\mathbb{X} = \{0, 1\}\) (presence or absence of
properties). Ordering does not matter.
\item \textbf{Ordinal scale}: Measurement values are meaningful only with respect to other measurements, i.e., the rank order of measurements carries
the information, not the numerical differences {\color{orange}(\emph{e.g. information
on the ranking of different marathon races)}}
\item \textbf{Quantitative scales}
\begin{itemize}
\item \textbf{Interval scale}: The relation of numerical differences carries
the information. Invariance w.r.t. translation and scaling {\color{orange}\emph{(Fahrenheit scale of temperature)}}.
\item \textbf{Ratio scale}: Zero value of the scale carries information but
not the measurement unit. {\color{orange}\emph{(Kelvin scale)}}.
\item \textbf{Absolute scale}: Absolute values are meaningful. {\color{orange}\emph{(grades of final exams)}}
\end{itemize}
\end{itemize}


\end{document}