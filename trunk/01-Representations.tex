\documentclass[MachineLearning]{subfiles}
\begin{document}
\setcounter{section}{1}
\section{Representations}

%@@@@@@@@@@@@@@@@@@@@@@@@@@@@@@
% summarizes lecture 1 & 2
%author: Benjamin Ellenberger


\subsection{Measurements and Data}
How to represent data is a crucial part to successful learning. The concrete choice of representation ('features') Learning methods expect standardized representations of data (e.g. Points in vector spaces, nodes in a graph, similarity matrices ...)
\subsubsection{Taxonomy of data}
\textbf{Goal:} We like to represent objects/items of interest and
characterize them according to their typical patterns for detection, classification, abstraction (compression), $\ldots$.

\begin{itemize}
\item Object space: Design/Configuration/ Object space \(\printlatex{\mathcal{O}}\)
\item Measurement/Feature: Measurement X maps an object into a domain 
\(\printlatex{X: \mathcal{O}^{(1)}\ldots\mathcal{O}^{(R)}\rightarrow \K}\) and represents an object in a data space.
\item Data
\begin{itemize}
\item \textbf{Monadic data} \(\printlatex{\mathcal{O} \rightarrow \R^{d}}\): An indivisible, impenetrable unit of data such as temperature, water depth, pressure, intensity etc..
\item \textbf{Dyadic data} \(\printlatex{\mathcal{O}^{(1)}\times \mathcal{O}^{(2)} \rightarrow \R}\): Data consisting of two monadic datasets of different object spaces. This can be \(\printlatex{\{users\} \times \{websites\}}\) etc..
\item \textbf{Polyadic data} \(\printlatex{\mathcal{O}^{(1)}\times \mathcal{O}^{(2)} \times \ldots \times \mathcal{O}^{(R)} \rightarrow \R}\): Data consisting of multiple monadic datasets of different object spaces. 
\item \textbf{Pairwise data} \(\printlatex{\mathcal{O} \times \mathcal{O} \rightarrow \R}\): Data that comes in pairs of the same object space such as \(\printlatex{\{proteins\} \times \{proteins\}}\) comparisons.
\end{itemize}
\end{itemize}

The choice of features X is an engineering problem. We must find a tradeoff between underfitting and overfitting the data.
\begin{enumerate}
\item Ideally as informative as possible about label Y
\item Must consider cost of acquiring / computing them
\item Poor choice of features \(\printlatex{\rightarrow}\) no luck with learning!
\end{enumerate}
\subsection{Data Types, Scale, Transformation invariances}
\subsubsection{Data Types and Scale}
\begin{itemize}
\item \textbf{Nominal or categorical scale}: Qualitative, but without quantitative measurements, e.g. binary scale \(\printlatex{\mathbb{X} = \{0, 1\}}\) (presence or absence of
properties). Ordering does not matter.
\item \textbf{Ordinal scale}: Measurement values are meaningful only with respect to other measurements, i.e., the rank order of measurements carries
the information, not the numerical differences {\color{orange}(\emph{e.g. information on the ranking of different marathon races)}}
\item \textbf{Quantitative scales}
\begin{itemize}
\item \textbf{Interval scale}: The relation of numerical differences carries
the information. Invariance w.r.t. translation and scaling {\color{orange}\emph{(Fahrenheit scale of temperature)}}.
\item \textbf{Ratio scale}: Zero value of the scale carries information but
not the measurement unit. {\color{orange}\emph{(Kelvin scale)}}.
\item \textbf{Absolute scale}: Absolute values are meaningful. {\color{orange}\emph{(grades of final exams)}}
\end{itemize}
\end{itemize}
\subsubsection{Transformation invariances}
\textbf{Importance of invariances:} if the measurements are invariant under a set of transformation then the mathematical definition of structure should obey the same invariances. Otherwise, our structure search procedure breaks the symmetry in an a priori (not data-dependent) way.
\begin{table}[H]
\centering
\begin{tabular}{|l|l|}
\hline
scale type & transformation invariances \\
\hline
nominal & \(\printlatex{T = \{f : \R \rightarrow \R~|~\text{f bijective}\}}\)\\
ordinal & \(\printlatex{T = \{f : \R \rightarrow \R~|~f(x_1 ) < f (x_2 ), \forall x_1 < x_2 \}}\)\\
interval & \(\printlatex{T = \{f : \R \rightarrow \R~|~f(x) = ax + c, a \in \R^+ , c \in \R\}}\)\\
ratio & \(\printlatex{T = \{f : \R \rightarrow \R~|~f(x) = ax, a \in \R^+ \}}\)\\
absolute & \(\printlatex{T = \{f : \R \rightarrow \R~|~\text{f is identity map}\}}\)\\
\hline
\end{tabular}
\caption{Formal characterisation of different scale types and their invariance properties.}
\end{table}

\subsection{Readings}
\todo{Write down readings from the books covering the topics of the representations section.}

\end{document}